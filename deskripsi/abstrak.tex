% Mengubah keterangan `Abstract` ke bahasa indonesia.
% Hapus bagian ini untuk mengembalikan ke format awal.
\renewcommand\abstractname{Abstrak}

\begin{abstract}

  % Ubah paragraf berikut sesuai dengan abstrak dari penelitian.
  \par Helm keselamatan kerja merupakan salah satu APD  yang berfungsi untuk melindungi kepala dari segala bentuk hantaman langsung ke kepala penggunanya. Terdapat aturan pemerintah yang mengatur dan mewajibkan penggunaan Alat Pelindung Diri dimana salah satunya yaitu Helm Keselamatan Kerja seperti pada Peraturan Menteri Tenga Kerja dan Transmigrasi Republik Indonesia NOMOR PER.08/MEN/VII/2010 tentang ALAT PELINDUNG DIRI. Tetapi hal itu tidak menjamin semua pekerja mengenakan Helm Keselamatan Kerja walaupun sudah ada instruksi untuk digunakan. Masih sering didapati pekerja yang mengabaikan keselamatan kerja. Perusahaan - perusahaan yang melakukan pekerjaan pada umumnya sudah mengerahkan pengawas yang bertugas untuk mengawasi penggunaan APD dimana pengerahannnya juga sudah ada aturan yang mengatur. Tetapi pengawas masih melakukan pengawasan secara manual yang dimana memiliki limitasinya sendiri seperti luas area yang dapat diawasi dan banyaknya jumlah pekerja yang harus diawasi. Maka dari itu, dalam penelitian ini diambil suatu tujuan yaitu Merancang sistem yang dapat mendeteksi penggunaan helm keselamatan kerja secara \emph{real-time}. Setelah dilakukan pengujian, didapatkan hasil pengujian model dengan nilai \emph{precision} 0,92, \emph{recall} 0,87, mAP@.5 0,9 dan untuk pengujian sistem \emph{trigger alarm} didapatkan akurasi paling rendah 0,82.
\end{abstract}

% Mengubah keterangan `Index terms` ke bahasa indonesia.
% Hapus bagian ini untuk mengembalikan ke format awal.
\renewcommand\IEEEkeywordsname{Keywords}

\begin{IEEEkeywords}

  % Ubah kata-kata berikut sesuai dengan kata kunci dari penelitian.
  Visi Komputer, \emph{You Only Look Once} (YOLO), Helm Keselamatan Kerja

\end{IEEEkeywords}
