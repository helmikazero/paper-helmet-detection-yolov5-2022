% Mengubah keterangan `Abstract` ke bahasa indonesia.
% Hapus bagian ini untuk mengembalikan ke format awal.
\renewcommand\abstractname{Abstrak}

\begin{abstract}

  % Ubah paragraf berikut sesuai dengan abstrak dari penelitian.
  Keselamatan dan Kesehatan Kerja bertujuan meningkatkan standar dan kualitas kerja di era modern ini. Pengaplikasiannya pun beragam, salah satunya yaitu penerapan penggunaan helm keselamatan kerja atau helm proyek atau Hard Hat. Medan proyek konstruksi yang berisiko tinggi menjadi alasan utama pekerja proyek harus benar - benar mematuhi aturan penggunaan APD, dimana salah satunya penggunaan helm proyek. Helm proyek membantu mengurangi dampak benturan misal saat pengguna terjatuh atau tertimpa benda berat atau tajam. Tetapi tidak semua personel lapangan akan dengan sendirinya mematuhi aturan ini sehingga diperlukannya pengawasan dalam penerapan penggunaan helm proyek sebagai salah satu APD keselamatan kerja. Pengawas atau supervisor lapangan yang dikerahkan adalah personil manusia yang juga memiliki batasannya sebagai manusia. Kondisi lapangan proyek yang luas dan banyaknya personil lapangan akan menjadi suatu kesulitan untuk pengawas manusia untuk memastikan tiap personil lapangan mematuhi aturan penggunaan helm keselamatan kerja. Maka dari itu, dalam penelitian ini diambil suatu tujuan yaitu merancang sistem yang dapat mendeteksi penggunaan helm proyek secara otomatis. Dalam perancangan sistem ini, akan memanfaat Convolutional Neural Network yang didesain untuk rekognisi data dua dimensi. Sistem yang sudah jadi akan diuji pada lapangan proyek konstruksi.

\end{abstract}

% Mengubah keterangan `Index terms` ke bahasa indonesia.
% Hapus bagian ini untuk mengembalikan ke format awal.
\renewcommand\IEEEkeywordsname{Kata kunci}

\begin{IEEEkeywords}

  % Ubah kata-kata berikut sesuai dengan kata kunci dari penelitian.
  \emph{You Only Look Once} (YOLO), Visi Komputer,Helm Keselamatan Kerja.

\end{IEEEkeywords}
