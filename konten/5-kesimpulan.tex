% Ubah judul dan label berikut sesuai dengan yang diinginkan.
\section{kesimpulan}
\label{sec:kesimpulan}

\par Berdasarkan pengujian yang dilakukan pada penelitian ini, ditarik beberapa kesimpulan sebagai berikut:

\begin{enumerate}[nolistsep]

    \item YOLOv5 dapat digunakan untuk deteksi helm keselamatan kerja dibuktikan pada pengujian masing - masing model dengan \emph{Pretrained Weights} dari YOLOv5 ataupun tanpa \emph{Pretrained Weights} untuk setiap varian (N,S,M,L) dimana mendapatkan rata - rata \emph{precision} 0,92, \emph{recall}  0,87 dan mAP@.5 0,92 yang dimana juga tidak ada perbedaan signfikan diantara varian.

    \item Pada pengujian kecepatan \emph{inference}, jika diurutkan dari paling cepat hingga yang paling lambat yaitu \emph{Nano}(N), \emph{Small}(S), \emph{Medium}(M), lalu \emph{Large}(L) dimana juga berpengaruh pada \emph{frame-rate} yang dibuktikan pada pengujian di Jetson Nano dimana varian N mendapatkan 18,4 FPShingga varian L mendapatkan 1,8 FPS

    \item Sistem deteksi helm keselamatan dapat dilakukan pada pengujian jarak 1 meter hingga 10 meter dibuktikan melalui pengujian model mendapatkan nilai rata-rata pada semua jarak untuk \emph{precision} 0,9, \emph{recall} 0,97 dan mAP@.5 0,98  dan juga dari pengujian sistem alarm mendapatkan akurasi paling rendah 0.92.

    \item Sistem deteksi helm keselamatan kerja dapat dilakukan pada pencahayaan rendah dibuktikan melalui pengujian model dengan nilai rata-rata dari semua varian model untuk \emph{precision} 0,76, \emph{recall} 0,74 dan mAP@.5 0,78 dan juga dari pengujian sistem alarm dengan varian \emph{Small} mendapatkan nilai 0,82 dimana menunjukkan terjadi penurunan performa pada pencahayaan rendah.

    \item Sistem deteksi helm keselamatan kerja dapat dijalankan secara \emph{real-time} ditunjukkan pada pengujian sistem di Jetson-Nano dengan akurasi 0.92.

    \item Menimbang hasil pengujian dari semua model yang dimana perbedaannya tidak terlalu signifikan untuk kecepatan \emph{inference}-nya, varian model YOLOv5s dengan \emph{weight} dengan \emph{Pretrained Weight} varian \emph{Small} dianggap sebagai pilihan paling optimal pada keperluan deteksi helm keselamatan kerja secara \emph{real-time} yang lalu juga digunakan untuk semua pengujian sistem.
    
\end{enumerate}

