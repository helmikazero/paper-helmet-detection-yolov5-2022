% Ubah judul dan label berikut sesuai dengan yang diinginkan.
\section{Introduction}
\label{sec:introduction}

% Ubah paragraf-paragraf pada bagian ini sesuai dengan yang diinginkan.

\par Safety helmet or hardhat is one of the Personal Protective Equipment or commonly refered as "PPE".
 The Republic of Indonesia Ministry of Manpower have already regulated the use of safety helmet
 in NOMOR PER.08/MEN/VII/2010 for Regulation of Personal Protective Equipment that stated the obligation
 of several equipments listed as Personal Protective Equipment in which safety helmet included as "head protective gear" \cite{kementrianpekerjaanumum}.
 In general, safety helmet protects the wearer from any form of impact directed to the head of the wearer.

 \par Being regulated in goverment regulation does not guarantee that every worker
 in the field wear safety helmet when instructed to. General Secretary of National Construction Services of Indonesia
 (GAPENSI) stated that several constructions that was being worked on by State-owned Enterprise of Indonesia
 had its worker caught not wearing the hardhat. 

 \par In overcoming negligence in the use of safety helmets or other 
 Personal Protective Equipment (PPE), companies that carry out the construction, in general, have mobilized 
 supervisors or supervisors in the form of OSH officers or OSH experts 
 who are also tasked with supervising the use of Personal Protective Equipment
 as a form of Occupational Safe and Health. 
 In addition, the deployment itself is regulated in REGULATION OF THE 
 MINISTER OF PUBLIC WORKS NUMBER: 05/PRT/M/2014 Article 6 paragraphs 
 1 and 2 states that it is obligatory to involve OSH experts or 
 officers in low or high potential hazards \cite{suratkementriantenagakerja}. However, 
 the OSH officers deployed in general still carry out manual 
 supervision. Here, it is known that humans have certain 
 limitations where the area of supervision is too broad and the 
 number of workers to be supervised is a challenge.

 \par Apart from the existence of Occupational Health and Safety rules and the deployment 
 of supervision with all its limitations, based on the Accident 
 and Occupational Disease Data for the Second Quarter of 2020 
 from the Ministry of Manpower, Type A work accidents which 
 include "A collision (to the head) generally indicates 
 contact with a sharp object or hard object that causes 
 it to be scratched, cut , stabbed, etc" reached 878 accidents 
 which is the largest number compared to other types of 
 accidents with Type J (others) with 637 and Type C (squeezed) 
 with 439 \cite{satudata_kecelakaan_kerja}.