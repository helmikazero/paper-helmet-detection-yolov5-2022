% Ubah judul dan label berikut sesuai dengan yang diinginkan.
\section{Related Studies}
\label{sec:relatedstudies}

% Ubah paragraf-paragraf pada bagian ini sesuai dengan yang diinginkan.
There are several studies that have been 
done previously related to this research, 
such as the Deep Learning Based Safety 
Helmet Detection in Engineering Management 
Based on Convolutional Neural Networks 
which was conducted by Li and friends 
in 2020 which is about the method of 
detecting safety helmets in real-time 
based on deep learning on construction 
sites. Li and friends use SSD-MobileNet 
which is based on CNN. Using a dataset 
of 3261 images of safety helmets. 
SSD-Mobilenet was chosen over R-CNN 
with the intention of faster detection 
and is suitable for real-time even 
though it is not as accurate as 
R-CNN. \cite{li2020deep} Then \emph{Deteksi Penggunaan Helm 
Pada Pengendara Bermotor Berbasis} Deep Learning by Yusuf Umar in 2020 who 
conducted research on the detection of 
helmet use on motorcyclists. To conduct this research, Hanafi used YOLOv3 
in which is also based of CNN. 
In the developed system, it can provide 
a bounding box to the rider, then in the 
rider's bounding box, there is another 
bound box from the rider's head to the 
chest to detect the use of a motorcycle 
helmet or not. \cite{hanafi2020deteksi} 
And also Safety Helmet Detection Based on YOLOv5 by Zhou and friends in early 
2021 in the form of a work safety helmet detection research based on YOLOv5. 
In their research, Zhou and friends did a comparison with 4 models from YOLOv5 
which include YOLOv5s, YOLOv5m, YOLOv5l, and YOLOv5x. In addition, Zhou and 
friends used a dataset containing 6054 images collected from the internet 
and annotated themselves. There are two annotation labels, namely "Alarm" 
which is a head without a helmet, and "helmet" which is a head with a work 
safety helmet\cite{zhou_zhao_nie_2021}.