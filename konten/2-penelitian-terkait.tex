% Ubah judul dan label berikut sesuai dengan yang diinginkan.
\section{Penelitian Terkait}
\label{sec:penelitianterkait}

% Ubah paragraf-paragraf pada bagian ini sesuai dengan yang diinginkan.
Terdapat beberapa penelitian yang sudah dilakukan sebelumnya yang berkaitan dengan penelitian ini yaitu seperti
\emph{Deep Learning Based Safety Helmet Detection in Engineering Management Based on Convolutional Neural Networks} yang dilakukan oleh
Li dan teman teman pada tahun 2020 tentang metode deteksi helm keselamatan kerja secara real time berbasis deep learning pada lokasi konstruksi. Li dan teman – teman menggunakan SSD-MobileNet yang berbasis dari CNN. Menggunakan dataset yang berjumlah 3261 gambar helm keselamatan. SSD- Mobilenet dipilih dibanding R-CNN dengan maksud pendeteksian yang lebih cepat dan cocok untuk real – time walau tidak seakurat R-CNN. \cite{li2020deep}
Lalu Deteksi Penggunaan Helm Pada Pengendara Bermotor Berbasis Deep Learning 
oleh Yusuf Umar pada tahun 2020 yang melakukan penelitian tentang deteksi penggunaan helm pada pengendara bermotor. Pada penelitiannya menggunakan YOLOv3 yang berbasis dari CNN. Pada sistem yang dikembangkan dapat memberikan bounding box ke pengendara lalu dalam bounding box pengendara terdapat boundbox lain dari kepala hingga dada pengendara untuk mendeteksi penggunaan helm motor ada atau tidak. \cite{hanafi2020deteksi}
Dan juga \emph{Safety Helmet Detection Based on YOLOv5} oleh Zhou dan teman - teman pada awal tahun 2021 yang berupa penelitian deteksi helm keselamatan kerja yang berbasis dari YOLOv5. Pada penelitiannya Zhou dan teman - teman melakukan perbandingan dengan 4 model dari YOLOv5 yang meliputi YOLOv5s, YOLOv5m, YOLOv5l, dan YOLOv5. Selain itu Zhou dan teman - teman menggunakan dataset yang berisi 6054 gambar yang dikumpulkan dari
internet dan di anotasi sendiri. Label anotasinya ada dua yaitu "Alarm" yang merupakan kepala tanpa helm dan
"helmet" yang merupakan kepala dengan helm keselamatan kerja\cite{zhou_zhao_nie_2021}.